\chapter{Conclusion}
\label{chap:final}

Here are some final comments about putting together a thesis.  They
are mostly opinion (and certainly opinionated) so take on board what
you will.  This is free advice after all (and may be worth as much)
but it is meant to make life easier for you.

\section{10 Things You Need to Believe \ldots}
\subsection{Use Sections}
Chapters are BIG things.  Write an opening paragraph to your chapter,
then mark in each section you are going to cover.  Fill in the actual
content AFTER you have worked out just what the sections are going to
be.

\subsection{Do your Literature Review First}
Lots of people do their literature review after they have written
their program/done their experiment.  Don't fall into this trap---you
think the write-up will only take six weeks, but it won't, because
the lit.\ review will take four.  Create a BibTeX file while you do
the review, so you can cite as you go.  You would be amazed at how
many people don't do this.

\subsection{Do Not Write a Diary}
Nobody is interested in how you went about writing your program.  The
academic community is interested in how you have integrated the
existing theory with your own ideas, and the department is interested
in why you made particular design decisions.  Remember that we are a
Humanities department as well as a Science department, so we want to
see some sort of convincing argument for your ideas.

\subsection{Avoid Visual Formatting}
Try not to use commands like \verb|\vspace, \pagebreak| or
\verb|\enlargethispage|.  The whole point of \LaTeX\ is that it
provides a markup language that works perfectly 90\% of the time.
This means that when you have {\bf finished}, the last thing you
should do is print up a draft, then go through and mark any formatting
you don't like (there will probably be about one thing every ten
pages).  That is the time to insert things like \verb|\pagebreak|
commands.

\subsection{Learn \LaTeX\ Early}
The learning curve for \LaTeX\ is steep but short.  The idea is that
some poor sod like me does all the dirty work, and all you have to do
is {\em fill in the content}.  However you may spend about a week
learning all the fiddly bits (like tables and figures), and you don't
want to be taking time away from the actual thesis writing.

\subsection{Read Some Documentation}
In the Systems Lab, we are going to make every attempt to have
printed documentation lying around.  If you need quick help, check out
the file {\tt essential.dvi} on any Linux \LaTeX\ installation.  It
is probably the best short guide to \LaTeX\ around, and even contains
lots of tricky maths examples.

\subsection{Save Paper}
On a mac, use the layout options to print two-up and double-sided. On a
linux machine, use {\tt  psutils} to print your drafts.  This way you
use a quarter of the paper.  Here's what to do if you don't have
a duplex printer:
\begin{enumerate}
\item Use the twosided option and make the PDF file.
\item Use \verb|pdftops| to convert \verb|thesis.pdf| to \verb|thesis.ps|.
\item Use {\tt psbook} to arrange the pages for booklet printing;
\begin{quote}
e.g. {\tt psbook thesis.ps thesis.bk.ps}.
\end{quote}
\item Use {\tt psnup} to get the book to a booklet;
\begin{quote}
e.g. {\tt psnup -2 thesis.bk.ps thesis.2up.bk.ps}.
\end{quote}
\item Use {\tt psselect} to print up the even pages first (in reverse),
then the odd pages;
\begin{quote}
e.g.\\
type {\tt psselect -e -r thesis.2up.bk.ps | lpr}\\
Put the result into the sheet feeder, blank side up\\
type {\tt psselect -o thesis.2up.bk.ps | lpr}\\
\end{quote}
\end{enumerate}

That's the theory; in practice it goes something like this:
\begin{quote}
{\tt psbook thesis.ps | psnup -2 | psselect -e -r | lpr}\\
Now open the laserjet side-door and put the paper which came out after
the first command on the tray.  Don't change its orientation or
anything: just pick it up, move it to the tray and drop it.  Then:\\
{\tt psbook thesis.ps | psnup -2 | psselect -o | lpr}\\
\end{quote}

\subsection{Use \LaTeX, not Word}
Microsoft Word is not your friend.  Word is not {\em anybody's} friend.
It is possible to write large documents in Word, but you have to be
{\bf so} strict on yourself (using heading styles, TOC entries, etc.)
that you may as well have used \LaTeX\ in the end anyway.  Once a
problem is nailed in \LaTeX, it stays nailed.  The same is not true of
Word.  Pdflatex produces standard PDF which can be
printed anywhere, and is fast becoming a standard for on-line article
publication.  Ask anyone how many problems they have had printing Word
documents to PostScript printers.

\subsection{The First Copy is Not the Final Copy}
Well, unless you're Isaac Asimov, or Mozart. Print up
draft copies of chapters and give them to your supervisor to read,
then implement any changes when they get returned.  Then, a week
later, reread the chapter and change anything that makes you cringe.

\subsection{Spell Check your Work}
If you are using UNIX (as I have assumed throughout this document)
then the {\tt ispell} program is a pretty good spell checker.  Also
get someone to check your punctuation and grammar, and be on the
lookout for spellchecker errors (like ``fro'' instead of ``for'',
which will not be picked up at all).  Even the {\em worst} writer of
novels who gets published has a good grasp of grammar, because it is
unpleasant to read work which doesn't make sense.  There is no point
in putting the examiner in a bad mood by turning the reading of your
thesis into an unpleasant chore.

\section{General Comments}
Traditionally, academic works are written using the passive
voice---``this was done'' rather than ``we did this''.  Also, the use
of the personal pronoun ``I'' is avoided in favour of ``we''; but here
you must be careful.  There are two ways of using ``we'': either {\em
in}clusively or {\em ex}clusively.  Inclusively is pretty much alright
as it makes the reader feel like part of the story; e.g. ``Reducing
Equation 2.3 to Equation 2.4, we can see \ldots''.  Exclusively sounds
pompous---``we now present a new algorithm which will solve this
problem \ldots'' and should be avoided unless writing a paper with
more than one author.

There are some other things worth remembering:
\begin{itemize}
\item When referring to another chapter or section, Use a capital
letter; e.g. ``see Chapter~3'' not ``see chapter~3''.  Use a tilde
character $(\sim)$ to put a non-breaking space before the number.  That
way you will never begin a new line with the number.
\item The layout suggested in this document (introduction, lit.\
review, new ideas, implementation, results, conclusion) is pretty
generic.  If you stick to it, you will complete a thorough but boring
thesis.  You will almost certainly need to digress occasionally, and
perhaps integrate the literature review more with your own work.
While academic writing isn't usually noted for its racey prose, it
doesn't have to be boring.
\end{itemize}

Finally, at the end of your thesis, don't be afraid to blow your own
trumpet.  If you have done something new and original, restate just
what that is.  The last part of the concluding chapter is what most
people read straight after the abstract, so it has to be just as pithy
and imagination-capturing.
